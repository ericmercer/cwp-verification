Methods for the design and verification of interactive health IT (HIT) have lagged far behind, and the complexity means systems may demonstrate behaviors that risk patient safety. This research intends to mitigate that risk by formalizing the relationship between the HIT and its intended work. It proposes that the work be described as a conceptual work product (CWP): a declarative model of what needs to be accomplished with required resources. It then proposes that the HIT be modeled in MATHflow: a variant of Business Process Modeling Notation with resources. Model checking is then employed to prove the MATHflow model implements the CWP under nominal human-computer interaction. The model checking step is accomplished using the SPIN model checker, and this paper presents a systematic translation of MATHflow to Promela. The CWP is directly expressed as a never claim. The entire process is illustrated on a patient contact system, and it suggests the model checking step effective and that much of it can be automated.
