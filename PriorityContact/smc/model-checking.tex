Model checking proves a transition system implements a
specification. To date, much of the research with model checking in
health care has focused on formalizing work flows and proving temporal
properties on work flows \cite{avrunin, clark, baski} or on developing new modeling
languages to describe work-flows and human machine teaming in those
work-flows \cite{mathflow, boulton}. The approach in this paper shifts
focus to the work intended to be accomplished by a work-flow. It uses
model checking to prove that a work-flow accomplishes the work
declared in the CWP.

Novel in the approach is the direct use of the CWP as the
specification for the model checking step using the SPIN model checker
\cite{spin}. The CWP as the specification for model checking is in
direct contrast with prior work that would rather restate the CWP as a
temporal logic specification. Using the CWP directly as the property
to verify in model checking separates the verification process into
two steps: first, verify that the CWP describes the intended
work; and second, verify that the work-flow accomplishes the
intended work. The rest of this section describes the second step.

The input to the verification process is the MATHFlow model of the
work-flow with the CWP. The CWP consists of two components: the
ontology describing the state of the CWP, and the state chart
describing the legal evolution of that state. It is assumed that the
MATHFlow model uses a well-defined interface to interact with the
state of the CWP. In particular, that MATHFlow model must delineate
where and how it mutates the state of the CWP. With that clear
interaction defined, the verification process proceeds as follows:

1) The MATHFlow model is analyzed to extract the interaction with the CWP.
2) The CWP ontology is abstracted to the defined interaction in the
   MATHFlow model and expressed as a data structure in SPIN.
3) The MATHFlow model is analyzed to extract interaction between
   actors. Communication channels or shared variables for communication
   are created to support those interactions.
4) The MATHFlow model is translated into Promela, the SPIN modeling
   language, with each actor interacting with the data-structure and communication
   channels as appropriate.
5) The CWP state chart is directly expressed as a never claim in Promela.
6) SPIN model checks the never claim looking for violations.

Steps (1) and (2) of the verification process are the most
complex. The CWP ontology defines all the information related to the
CWP with its intra-relations, but not all that information, nor its
intra-relations, indicate the state of the CWP or are directly
utilized in the MATHFlow model. For example, the ontology for the
priority contact model has a notion of a \emph{Contact Plan} that
includes not only key attributes that define part of the actual state of the contact plan such as
\emph{Prioity}, \emph{Launch Time}, and \emph{Resolve Time}, but also
doctor information, patient information, disease, treatment plan,
etc. that is necessary information to accomplish the \emph{Contact
  Plan}. All of this auxiliary information is read-only, and does not
distinguish the actual state of the CWP at any given point in time and
as such that information, with its various relationships to other
information, can be abstracted in the verification model.

Continuing with the example of the priority contact ontology, the
MATHFlow model of the priority contact only cares if a lab result
indicates that the diagnosis is life threatening, life changing,
routine, or no change relative to the patient's life. As such, the
data structure representing the ontology in the Promela verification
model can safely abstract anything related to the disease, including
the symptoms and treatment plan, to just a single field to indicate
one of the four previously mentioned categories: life threatening,
life changing, routine, or no change.  As a further example, although
the CWP ontology includes all the information necessary to contact the
patient, the actual name of the patient or contact information, is not
important in the MATHFlow description, so it is omitted from the data
structure representing the CWP ontology. In this way, steps (1) and
(2) abstract the CWP ontology to two parts: fields that define the
state of the CWP and are mutated by the MATHFlow description; and
fields that define input to the MATHFlow description that affect the
actual flow (e.g., a life threatening lab result follows a different
process than a routine lab result). The process of creating the
ontology abstract from the MATHFlow model is difficult to automate and
expected to be a very manual exercise for the modeler.

Step (3) relates to the different actors in the MATHFlow
model. Each of these actors is a process in SPIN, and as processes,
operate concurrently. When two actors synchronize in a communication,
for example, in a life threatening diagnosis, an attempt to contact
the patient via phone is made, and if that fails, the patient is sent
texts and voice mails, as are the emergency contacts, until the
patient is found. This synchronization is modeled either with a
communication channel in Promela, or a shared variable that is
accessed by the parties involved in the communication. The choice of
communication is still an active area of research in the model
generation, but the important aspect is that the communication can
either be asynchronous (e.g., a text is sent to the patients phone) or
synchronous (e.g., the patient answers the phone). In either case, the
communication is identified and defined in step (3).

Step (4) is readily mechanized as the translation from MATHFlow to
Promela is nearly one-to-one as each actor is modeled with a process
in Promela and that process is defined as a state machine. The state
machine description has a label marking the state, a step of actions
that take place in that state, and goto statements moving to the next
state in the MATHFlow as appropriate. The actions either read input
from the CWP ontology data-structure (e.g., is this life-threatening,
etc.), update the state of the ontology (e.g., set the priority to
high or mark the contact as launched), or communicate with another
actor. Step (4) is somewhat trivial by virtue of steps (1) through (3)
which have done all the heavy lifting in the abstraction of the
ontology and the definition of the communication.

As a matter of completeness, the MATHFlow model does include
non-determinism from the severity of the diagnosis to whether or not a
patient is able to answer the phone immediately. This non-determinism
is included in the Promela model, and SPIN explores all possible ways
to resolve the non-determinism. However, the actors always behave in a
nominal way. For example, if the patient receives a text, then the
patient follows the instructions in the text.

Step (5) turns the actual state chart in the CWP to a never claim in
SPIN. The never claim in SPIN represents a finite state automaton that
is synchronously composed with the models of the actors. A synchronous
composition between the never claim and the actors means that when any
actor takes a step (i.e., updates its state), then the never claim
must agree that such an update is allowed. If the update is not
allowed, then the search terminates along that path and another actor
is selected to update. The CWP state-chart as a never claim accepts
all transitions that do not affect the actual state of the CWP. If the
state of the CWP is changed on a transition, then that transition is only accepted if it
follows a defined edge in the state chart. For example, from the
initial state of the state-chart, it is valid to move to the launched
state when a launch time is defined.

If a transition modifies the CWP in a way that does not follow an edge
on the state-chart, then the never claim moves to an error state. The
error state captures all the transitions made by the MATHFlow model
that are not allowed by the state-chart definition. For example, if
the MATHFlow model moves the CWP ontology from the start state
directly to the resolved state on some input and resolution of
non-determinism, then an error is flagged. As a final matter of
completeness, the never claim marks specific states as \emph{end}
states. All executions must leave the never claim in a valid end state
or an error is flagged.

Step (6) is the actual verification in the SPIN model checker. This
step may or may not be tractable depending on the level of concurrency
in the model (e.g., the number of actors doing actual work at the same
time) and the level of non-determinism in the read-only part of the
CWP ontology (i.e., input) and the non-determinism in the actual
different actors of the MATHFlow model. The priority contact model is
trivial for the SPIN model checker (less than a second to verify), and
the model checker itself has proven able to manage millions of
different states in the verification process. It is yet to be
determined if it is sufficient for an interesting class of CWP and
MATHFlow models to be of value.

