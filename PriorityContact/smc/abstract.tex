Rigorous methods for the design and verification of health IT (HIT) have lagged far behind its proliferation. The inherent complexity of health care embodied in HIT means that the system may contain unintended and unpredictable emergent behavior that poses risk to patient safety. The work presented in this paper intends to help mitigate such risk by formalizing the relationship between the HIT and its intended work. Specifically, this paper proposes that the intended work in HIT can be formally described as a conceptual work product (CWP). A CWP is a declarative description of the work intended to be accomplished by a system including the resources needed to accomplish that work. This paper then proposes that the actual process used by HIT to accomplish the work be modeled in MATHflow, an extension of the Business Process Modeling Notation that includes resources needed by tasks. This paper then proposes the application of model checking to prove the MATHflow model of the HIT accomplishes the work declared in the CWP under nominal human-computer interaction. This model checking step forms a basis for comparing two HIT deployments. To show the viability of this new technique, this paper presents a case study on an HIT system to contact patients regarding lab results. This proof of concept implementation demonstrates that it is possible to use model checking to prove a MATHflow model implements a CWP. As such, it is believed that this new technique can evaluate the workflows of complex HIT to improve safety and reliability. 
